% 3.1.Setup.tex
%	Last update: 2022/02/07 F.Kanehori

%\newpage
\subsection{セットアップ機能}
\label{subsec:Setup}

セットアップの目的は、\cmnd{cmake}を使用してSpringehadライブラリをビルドする
ための環境を整えることである。
\begin{narrow}{\small
	元々は、Visual Studioの新バージョンが出る度に\file{.sln}などを
	提供するのを止めるために\cmnd{cmake}を導入するといった経緯があり、
	そのための前処理ツールとして作成されたものである。
	しかしその後、
	Windowsでは\file{.sln}を用いる方法をメインとすることとなったために
	\cmnd{cmake}を利用することはオプションとなった。
	unixでは\cmnd{cmake}を使うことが前提としており、
	\file{Makefile}のメンテは行なっていない。
	
}\end{narrow}

\medskip
\begin{Description}{スクリプトファイル}
	\Path{<topdir>\BS core\BS src\BS setup.bat} … {\small Windows}\\
	\Path{<topdir>/core/src/setup.sh} … {\small unix}
\end{Description}

\medskip
\begin{Description}{起動方法}
	\cmnd{setup.bat [options]} \Hskip{10pt}(Windows)\\
	\cmnd{setup.sh [options]} \Hskip{10pt}(unix)

	\begin{Opts}
	  \Item[t]{-c (--check)}{\Hskip{20pt}setup fileの整合性チェックのみを実行}
	  \Item[t]{-f (--force)}{\Hskip{20pt}無条件に再setupを実行}
	\end{Opts}
\end{Description}

\medskip
\begin{Description}{スクリプトの動作}
    次のpythonを使用する
    \begin{Table}[r][60pt]{ll}
	\Item{Windows}{%
		\Path{<topdir>\BS buildtool\BS win32\BS python.exe}があればそれを、
		なければデフォルトの\script{python}を使用する。
		どちらもなければメッセージを出して終了する。}
	\Item{Unix}{%
		デフォルトの\cmnd{python}を使用する。
		なければメッセージを出して終了する。}
    \end{Table}
\end{Description}

次の形式で実行スクリプトを起動する(optionsはそのまま引き継ぐ)。
\begin{narrow}
	\cmnd{python setup.py [options] python-path}
\end{narrow}

\begin{Description}{スクリプトファイル}
	\Path{<topdir>\BS core\BS src\BS setup.py} … {\small Windows}\\
	\Path{<topdir>\BS core\BS src\BS setup\_helpers.py} … {\small Windows}\\
	\Path{<topdir>/ core/src/setup.sh} … {\small Windows}\\
	\Path{<topdir>/core/src/setup\_helpers.sh} … {\small unix}
\end{Description}

\medskip
\begin{Description}{起動方法}
	\cmnd{python setup.py [options] python-path}
	\begin{Args}[r][60pt]
	    \Item[t]{python-path}{%
		pythonのパス(pythonがパスに入っていないことがあるから。)}
	\end{Args}
	\begin{Opts}
	    \Item[t]{-c (--check)}{\Hskip{40pt}setup fileの整合性チェックのみを実行}
	    \Item[t]{-f (--force)}{\Hskip{40pt}無条件に再setupを実行}
	    \Item[t]{-F (--Force)}{\Hskip{40pt}\tt{-f}と同様(\tt{swig}をclean buildする)}
	    \Item[t]{-s file}{\Hskip{40pt}setupファイル名(デフォルトは\file{setup.conf})}
	    \Item[t]{-d num}{\Hskip{40pt}devenvの選択番号(ユーザ入力なし---デバッグ用)}
	    \Item[t]{-R (--repository)}{\Hskip{40pt}local repositoryのディレクトリ名}
	\end{Opts}
\end{Description}

\medskip
\begin{Description}{必要なツール(変数\var{required}に設定)}
	\begin{tabular}{l@{\Hskip{20pt}}l}
		Windows	& \tt{python, devenv, nmake, swig, (cmake)}\\
		unix	& \tt{python, gcc, swig, (cmake), gmake, nkf}\\
	\end{tabular}
\end{Description}

\medskip
スクリプトの動作

\begin{Description}{Step 1. \tt{python}のバージョンチェック(現在はチェックしていない)}
 	\tt{python}はバージョン3以上でなければならない
	(バージョン2.7にも一応対応している)。
\end{Description}

\begin{Description}{Step 2. setupファイルがあるならば整合性をチェック}
 	整合性チェックの内容
 	\begin{enumerate}
	  \item 必須ツールが登録されているか
	  \item 登録されているパスは有効か(\cmnd{prog --version}などで検査)\\
		   \cmnd{devenv}の場合は\cmnd{vswhere}コマンドで調べる
		   (複数見つかる場合にも対応)。\\
		   OKならば、パスとバージョンを
		   \var{prog\_registered, vers\_registered}に設定する。
	  \item \file{CMakeLists.txt}が存在するならば
		   \file{CMakeLists.txt.dist}のバージョンと比較する。
	\end{enumerate}
\end{Description}

\begin{Description}{Step 3. 必要なツールのパスの現状の調査}
	必要なツールについて、\cmnd{where, vswhere, which}などを用いて現在の環境を調べる
	(実行できるものについては\var{prog\_scanned, vers\_scanned}に設定する)。\\
	devenvの探し方
	\begin{enumerate}
	  \item	\Color{red}{\bf{
		\Path{C:\BS Program Files (x86)\BS Microsoft Visual Studio}\Cont\\
	  	\Hskip{160pt}\Path{\BS Installer\BS vswhere.exe}\\
		を実行してインストール済みVisual Studioの情報を取得する。\\
		今後リリースされるVisual Studioについては
		この方法で取得できることを期待している。
		だめならば、
		\file{setup\_helpers.py}の中の\Func{try\_find\_devenv}を
		修正する必要がある。}}
	  \item	候補が複数見つかったらどれにするか選択させる(ユーザ入力)。
	  \item	もし\DQuote{'0' to try another one}が選択されたならば、
		\cmnd{devenv /?} を実行してみてバージョンを調べる
		(PATHに入っていることが前提\,!)。
	\end{enumerate}
\end{Description}

\begin{Description}{Step 4. 次のいずれかの場合以外はここで終了}
	\begin{itemize}
	  \item	Setupファイルがない
	  \item	必要なプログラムが実行できない
	  \item	Setupファイルの内容と現在の環境とに相違がある
	  \item	\file{CMakeLists.txt}がない
	  \item	\file{CMakeLists.txt}のバージョンが古い
	  \item	\tt{-f}/\tt{-F}オプションが指定されている
	  \item	必要なプログラムが揃っていない(setupファイルがない場合)
	\end{itemize}
\end{Description}

\begin{Description}{Step 5. \tt{-f}オプション指定がないときは処理を継続するか確認する}
\end{Description}

\begin{Description}{Step 6. ビルドパラメータの設定}
	\begin{tabular}{l@{\Hskip{10pt}}l}
	    \tt{'plat'} & マシンアーキテクチャから\\
	    \tt{'conf'} & Release\\
	    \tt{'vers'} & \var{vers\_scanned['devenv']}から … (Windowsのみ)\\
	\end{tabular}
\end{Description}

\begin{Description}{Step 7. パス情報の設定(実はこの情報は利用していない)}
	\begin{tabular}{l@{\Hskip{10pt}}l}
	    \tt{'sprtop'}  & Springheadツリーのトップディレクトリ\\
	    \tt{'sprcore'} & coreディレクトリ\\
	    \tt{'sprsrc'}  & srcディレクトリ\\
	    \tt{'sprlib'}  & pythonローカルライブラリのディレクトリ\\
	\end{tabular}
\end{Description}

\begin{Description}{Step 8. \cmnd{swig}のmake (unixのみ)}
	\var{prog\_scanned['swig']}が\tt{'NOT FOUND'}であるか、
	または\tt{-f}/\tt{-F}オプションが指定されたときに\cmnd{make}を実行する。
\end{Description}

\begin{Description}{Step 9. Setupファイルの作成}
	Setupファイルは3つのセクションからなる。\\
	\Hskip{10pt}\tt{[prog]} \Hskip{10pt} プログラムのパス情報\\
	\Hskip{10pt}\tt{[path]} \Hskip{10pt} その他のパス情報\\
	\Hskip{10pt}\tt{[data]} \Hskip{10pt} \tt{plat}(プラットフォーム),
			\ \tt{conf}(ビルド構成), \tt{vers}(toolset ID)
\end{Description}

\begin{Description}{Step 10. \file{CMakeLists.txt}がなければ作成}
	\file{CMakeLists.txt.dist}からコピーするだけ。
\end{Description}

% end: 3.1.Setup.tex
