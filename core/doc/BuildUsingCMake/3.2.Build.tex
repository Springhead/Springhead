% 3.2.Build.tex
%	Last update: 2021/06/09 F.Kanehori
%\newpage
\subsection{ビルド}
\label{subsec:Build}
\parindent=0pt

ライブラリのビルドについては特に説明することはありません。

\medskip
\bf{unixの場合}
\begin{narrow}
	ディレクトリ\BldDir へ移動してmakeコマンドを実行してください。
	ライブラリファイルは\SprTop{/generated/lib}に生成されます。

	インストール先を指定した場合(\KQuoteS \ref{subsec:SetInstallDirectory} 
	インストールディレクトリの設定\KQuoteE 参照)には、
	makeコマンドの代わりにmake installコマンドを実行してください。
	ライブラリファイル(及びヘッダファイル)は指定した場所にインストールされます。
\end{narrow}

\medskip
\bf{Windowsの場合}
\begin{narrow}
	ディレクトリ\BldDir へ移動して\Path{Springhead.sln}をVisual Studioで実行し、
	プロジェクトSpringheadをビルドしてください。
	ライブラリファイルは\\
	\SprTop{/generated/lib/<\it{arch}>}に生成されます。
	\tt{<\it{arch}>} はマシンのアーキテクチャに従い、
	\Path{win64}または\Path{win32}のいずれかです。

	インストール先を指定した場合(\KQuoteS \ref{subsec:SetInstallDirectory} 
	インストールディレクトリの設定\KQuoteE 参照)には、
	プロジェクトSpringheadのビルドに続けて
	プロジェクトINSTALLを(プロジェクトのみ)ビルドしてください。
	ライブラリファイル(及びヘッダファイル)は指定した場所にインストールされます。
\end{narrow}

%-------------------------------------------------------------------------------
\bigskip
\thinrule{\linewidth}
%
%\bf{\tt{INSTALL}時の注意}
%
%\begin{narrow}
%ヘッダファイルとライブラリファイルのインストール先を指定している場合、
%ターゲット\tt{INSTALL}を実行してもライブラリファイルが正しくインストールされない、
%またはインストールされたファイルの内容が不正、という現象が発生しています。
%申し訳ございません。
%続けてもう一度\tt{INSTALL}を実行していただければ、
%正しいライブラリファイルがインストールされます。
%\end{narrow}
%
%\thinrule{\linewidth}
%\bigskip
\medskip

\SprLib を使用したプログラムの実行時にDLLが見つからないというエラーが発生した
場合には、
\def\Width{.55\linewidth}
\begin{center}\begin{tabular}{l@{ --- }l}
	\hline
	32ビット環境のときは & \RBox{\SprTop{/dependency/bin/win32}} \\
	64ビット環境のときは & \RBox{\SprTop{/dependecny/bin/win64}と \\
				     \SprTop{/dependecny/bin/win32} の両方} \\
	\hline
\end{tabular}\end{center}
にパスを通してください。
Visual Studioから実行するときは、
プログラムのプロパティを開き、
[構成プロパティ]---[デバッグ]---[環境] に
``\,\tt{path=上記のパス}''とします。

% end: 3.2.Build.tex
