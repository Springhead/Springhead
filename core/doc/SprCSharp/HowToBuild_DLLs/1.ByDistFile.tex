% 1.ByDistFile.tex
%	Last update: 2021/09/15 F.Kanehori
\newpage
\section{配布されたソリューションファイルを使用する場合}
\label{sec:bydistfile}
\parindent=0pt

\medskip
\noindent
配布されているファイルツリーの中には次のソリューションファイルが含まれており、
Visual Studio を使用してビルドすることができます。

以下、Springhead Libraryツリーのトップディレクトリを\tt{<topdir>}で表すものとします。

\Vskip{-.3\baselineskip}
\begin{narrow}
	\Vskip{.7\baselineskip}
	ファイルの位置\hspace{20pt}\FILE{<topdir>/core/src/SprCSharp/}
	
	\Vskip{.5\baselineskip}
	\def\VS#1{Visual Studio #1}
	\def\PH{Physicsを含む最小}
	\begin{tabular}{l@{\hspace{20pt}}c@{\hspace{20pt}}l} \hline
		ファイル名	& バージョン& DLL構成 \\ \hline
		\tt{SprCSharp14.0.sln}	 & \VS{2015} & 全体構成 \\
		\tt{SprCSharp14.0PH.sln} & \VS{2015} & \PH 構成 \\ \hline
		\tt{SprCSharp15.0.sln}	 & \VS{2017} & 全体構成 \\
		\tt{SprCSharp15.0PH.sln} & \VS{2017} & \PH 構成 \\ \hline
		\tt{SprCSharp16.0.sln}	 & \VS{2019} & 全体構成 \\
		\tt{SprCSharp16.0PH.sln} & \VS{2019} & \PH 構成 \\ \hline
	\end{tabular}
\end{narrow}

\bigskip
適切なソリューションファイルを開いたら、
\tt{Install}をスタートアッププロジェクトに指定してビルドします。
ビルドに成功すれば、次のDLLファイルが生成されます。

\begin{narrow}
	\FILE{<topdir>/generated/bin/win\{64|32\}/SprCSharp.dll} \\
	\FILE{<topdir>/generated/bin/win\{64|32\}/SprExport.dll} \\
	\FILE{<topdir>/generated/bin/win\{64|32\}/SprImport.dll}
\end{narrow}

\bigskip
上記のDLLファイルがビルドされていれば、
サンプルプログラム\tt{SprCsSample}を実行することが可能です。
ソリューションのターゲット\tt{SprCsSample}をビルドしてください。

\small{\begin{tabular}{ll}
	(注) & Sampleという名称ですが実体はDLLの動作を確認するための
		テストプログラムです。
\end{tabular}}

% end: 1.ByDistFile.tex
